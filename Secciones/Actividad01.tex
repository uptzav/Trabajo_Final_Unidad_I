\section{Introdución} 
\vspace{12mm} %5mm vertical space


Los indicadores de desempeño son parámetros de medición de la actividad bibliotecaria. Su aplicación permite evaluar el rendimiento de la biblioteca y, por consiguiente, identificar sus logros y limitaciones en la prestación del servicio bibliotecario. Asimismo, su manejo proporciona información para la toma de decisiones y la asignación del presupuesto.
Los indicadores aquí señalados no son considerados como estándares que hay que cumplir, sino que actúan como un estímulo de mejora continua en la biblioteca y como un modo de subrayar las mejores prácticas. Su continuo manejo nos ayuda a priorizar los servicios bibliotecarios.
Podemos llevar a cabo la evaluación de la gestión bibliotecaria mediante la aplicación de indicadores y realizarla en diferentes períodos dentro de la misma biblioteca. También evaluamos valiéndonos de comparaciones entre bibliotecas, pero con extrema precaución, tomando en cuenta cualquier diferencia en la constitución de las mismas, o haciendo referencia a indicadores generales como la Norma ISO 11620 e interpretando los datos cuidadosamente. Debemos señalar que la norma no incluye indicadores para la evaluación del impacto de las bibliotecas en los usuarios o en la sociedad.



		

